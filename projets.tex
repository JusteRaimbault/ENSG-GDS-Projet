


%\input{../header.tex}
%\setbeamertemplate{headline}{}


\documentclass{beamer}
\usepackage{mathrsfs,xcolor,setspace,comment,centernot,listings,framed}
\usepackage[utf8]{inputenc}
\usepackage[T1]{fontenc}

% config du theme metropolis
\usetheme[progressbar=frametitle,block=fill, titleformat=smallcaps,sectionpage=progressbar,]{metropolis}




%\title{Analyse spatiale}
%\subtitle{Présentation des projets}
\title{Présentation des projets}

\author{Juste Raimbault et Yann Meneroux\medskip\\
%\texttt{paul.chapron@ign.fr}\\
\texttt{yann.meneroux@ign.fr}\\
\texttt{juste.raimbault@ign.fr}
}


\institute{LASTIG, Univ Gustave Eiffel, IGN-ENSG}


%\date{Cours Analyse Spatiale - DeSIGeo\\\smallskip
%07/03/2022
%21/11/2023
%[14?]/12/2024
%12/12/2025
%}
\date{ING3 - Filière Geo Data Science - UE2 Analyse de données%\\\smallskip

}


%definition de la couleur du texte dans la balise \alert{}
\definecolor{vertIGN}{HTML}{96C31E} % vert IGN %vrai valeur #97BE0D
\setbeamercolor{alerted text}{fg=vertIGN}

\definecolor{grisIGN}{HTML}{22292F} % Gris IGN tiré vers le noir 
\setbeamercolor{background canvas}{bg=grisIGN}


% code pour placer le log ENSG dans le bandeau de titre 
\makeatletter
\setbeamertemplate{frametitle}{%
  \nointerlineskip%
  \begin{beamercolorbox}[%
      wd=\paperwidth,%
      sep=0pt,%
      leftskip=\metropolis@frametitle@padding,%
      rightskip=\metropolis@frametitle@padding,%
    ]{frametitle}%
  \metropolis@frametitlestrut@start%
  \insertframetitle%
  \nolinebreak%
  \metropolis@frametitlestrut@end%
  \hfill
  \raisebox{-0.6ex}{\includegraphics[height=4ex,keepaspectratio]{../../../Resources/logoENSG_small.jpg}}
  \end{beamercolorbox}%
}
\makeatother




% logo ENSG première page 
\titlegraphic{\vspace{4cm}\flushright\includegraphics[width=2cm,height=2cm]{../../../Resources/logoENSG_big.png}} 



\begin{document}
\metroset{background=dark} % change background theme according to manual
\maketitle	





% projet: req git repo pour rendu: release

\begin{frame}{Objectifs des projets}

$\rightarrow$ Utiliser les concepts vus en cours pour répondre à une question de recherche ou opérationnelle (par exemple à laquelle vous êtes confrontés dans votre alternance).

\bigskip

$\rightarrow$ \textbf{Dimension spatiale} de la question nécessaire.

\bigskip

$\rightarrow$ Mobiliser certaines méthodes et outils vus en TP pour y répondre.

\bigskip

$\rightarrow$ Combiner les points de vue, sensibilités, compétences, et approches.

\end{frame}

\begin{frame}{Exemples de données}

\textit{Jeux de données à chercher selon la question posée (ou trouver un sujet en fonction des données à votre disposition); doivent être accessibles et gratuites, préférentiellement ouvertes.}

\bigskip

\begin{itemize}
	\item Données publiques ouvertes : IGN, INSEE
	\item Données volontaires : OpenStreetMap
	\item Données utilisées en TP
	\item Données utilisées dans votre alternance
	%\item Données restreintes ? \textit{Ne pas perdre trop de temps à chercher des données}
	\item \ldots
\end{itemize}


\end{frame}

\begin{frame}{Exemples de thèmes: géographie urbaine}

%\textbf{Socio/eco urbaine : }

\begin{itemize}
	\item Organisation socio-économique au sein des aires urbaines, et comparaison entre aires urbaines
	\item Test empirique du modèle monocentrique d'Alonso
	\item Déterminants des flux migratoires entre aires urbaines
	\item Etude spatialisée de la croissance de la population
	\item Structures socio-spatiales des marchés immobiliers
\end{itemize}


\end{frame}


\begin{frame}{Exemples de thèmes: transports/mobilité}

%\textbf{Transports/mobilité : }

\begin{itemize}
	\item accessibilité TC/VP dans les aires urbaines françaises
	\item impact d'un projet de transport comme le Grand Paris Express
	\item priorités d'aménagements ferroviaires en France : grande vitesse vs lignes locales ? Cadre régional ou européen ?
\end{itemize}

\end{frame}


\begin{frame}{Exemples de thèmes: réseaux spatiaux}

		%\textbf{Autres disciplines / thèmes \ldots} : 
		
		\begin{itemize}
			\item étude des propriétés de percolation de graphes aléatoires spatiaux
			\item particularités des réseaux spatiaux
			\item propriétés de réseaux d'infrastructures
		\end{itemize}				
		
\end{frame}

\begin{frame}{Exemples de thèmes: divers}
	
	\begin{itemize}
		\item Pratique du e-sport
		\item Dynamique financières dans le marché des devises
		\item Occurrences des catastrophes naturelles et lien avec le changement climatique
\end{itemize}		
	
\end{frame}


\begin{frame}{Années précédentes}
	\begin{itemize}
		\item Chômage et accessibilité en Ile-de-France
		\item Dynamiques électorales
		\item Prix de l'essence
		\item Ecoutes Spotify par pays
		\item Crimes et délits à la commune en France
		\item Feux de forêt aux Etats-Unis
		\item AirBnb à Paris
		\item Prix immobiliers
		\item \ldots
	\end{itemize}
\end{frame}

\begin{frame}{Modalités concrètes}

\begin{itemize}
	%\item Petits groupes de 2,3 ou 4, autant que possible avec compétences et sensibilités diverses.
	\item Petits groupes avec compétences diverses	
	\item Possibilité de lier le sujet à d'autres projets, votre alternance, etc.
	\item Produire des analyses quantitatives (minimum 3 méthodes, en partie spatialisées) : langage/logiciels au choix tant qu'ils sont libres (R, python, \ldots).
	\item Code source ouvert $\rightarrow$ dépôt github pour chaque groupe.
	\item Rendu sous forme de \textit{release} github : code et rapport (une quinzaine pages) dans le dépôt, au plus tard la veille des présentations.
	\item Evaluation par présentation de 15 minutes à tous lors de la dernière séance le 13 janvier (une partie de l'évaluation portera sur le retour des autres). % 22 décembre 
	\item Séances de travail en autonomie et séances encadrées.
	%\item Pas trop long / de temps à y passer : $\sim$ 10h par personne (moitié du temps de cours)
\end{itemize}




\end{frame}




\end{document}
